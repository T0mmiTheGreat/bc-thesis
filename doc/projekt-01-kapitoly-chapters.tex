\chapter{Úvod}

\todo{Úvod}


\chapter{Představení hry}

\section{Pravidla}
Hra je typu \uv{battle royale}. Jejím cílem je tedy eliminovat všechny protivníky a~zůstat poslední naživu. Hráči ovládají kruhové entity nazývané \uv{bubliny} (\uv{bubbles}) ve 2rozměrném labyrintu. 

Hra umožňuje hráčům svislý, vodorovný i~diagonální pohyb. Překážky v~labyrintu jsou vyznačeny polygony, skrz které hráči nemohou procházet. Kromě toho představuje hrací plocha obdélníkový box, který taktéž není možné opustit. Mimo pohybu je možné změnit stav bubliny tím, že hráč (manuálně) zmenší její velikost. Tato možnost existuje z~důvodu, aby bylo možné projít úzkými chodbami, pro které je bublina příliš velká. Důsledkem však je, že hráč takto ztratí body života (viz následující odstavec), nicméně nelze takto ztratit poslední bod života.

Hráči začínají s~určitým množstvím bodů života a~zůstávají ve hře, dokud je toto množství větší než 0. Když se 2 a~více hráčů dotýká, všichni dotýkající se hráči ztrácí body života. Kromě toho na této veličině závisí několik dalších vlastností hráče:
\begin{itemize}
    \item Velikost -- poloměr hráčovy bubliny (přímo úměrně),
    \item Rychlost pohybu (nepřímo úměrně),
    \item Síla -- kolik bodů života ubírá hráč svým protivníkům,
    \item Množství života získaného z~bonusů -- viz dále (nepřímo úměrně).
\end{itemize}

V~průběhu hry se v~náhodných intervalech objevují na hrací ploše bonusy ve tvaru čtverce, které hráčům doplňují body života. Aby byl bonus na hráče aplikován, musí se bonus s~bublinou hráče navzájem dotýkat. Následkem toho tento bonus zmizí a~žádný jiný hráč ho poté už nemůže použít. Místo, kde se bonus objeví je náhodné. Nikdy však nemůže kolidovat s~překážkou, ani se nemůže objevit v~určité vzdálenosti od nějakého hráče nebo jiného bonusu.

\section{Uživatelské rozhraní}

\todo{Vstup z klávesnice}
\todo{Výstup na obrazovku}
    \todo{Hrací plocha}
    \todo{Navigace}

\section{Editor herní plochy}

\section{Podobné hry}
\todo{Agar.io -- player model, contact damage, ...}
\todo{Worms -- bonuses, AI}


\chapter{Implementace}

\section{Použité knihovny}

\subsection*{Simple DirectMedia Layer}

SDL (Simple DirectMedia Layer) je multiplatformní knihovna, která poskytuje nízkoúrovňový přístup ke vstupu z~klávesnice, myši, joysticku, výstupu zvuku a~grafiky. Je napsána v~jazyce C a~funguje nativně i~v~C++. Oficiálně podporuje platformy Windows, macOS, Linux, Android a~iOS. V~projektu je použita verze 2 této knihovny (SDL2).

SDL má také několik přidružených knihoven, které rozšiřují její funkcionalitu. \todo{Satellite libs used in the project}

\subsection*{libSDL2pp}

Knihovna libSDL2pp slouží pro reprezentaci SDL2 a~přidružených knihoven pomocí objektového modelu jazyka C++.

\subsection*{SDL\_gfx}

Knihovna SDL\_gfx slouží pro vykreslení některých základních geometrických tvarů, které SDL v~základu nepodporuje, jako jsou například křivky a~elipsy. Je napsána v~jazyce C a~funguje nativně i~v~C++.


\chapter{Umělá inteligence ve hře}


\chapter{Závěr}

\todo{Závěr}
